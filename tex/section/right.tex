%---------------------------------------------------------------------------------------
%	TITLE  HEADER
%----------------------------------------------------------------------------------------
%---------------------------------------------------------------------------------------
%	PROFILE
%----------------------------------------------------------------------------------------
\cvsection{Biographie}
\vspace{4pt}

\cvtext{Ich habe Wirtschaftsinformatik an der Universität Regensburg studiert. Meine Studienschwerpunkte waren Internet Business und Informationssicherheit im Bachelor und Informationssicherheit im Master. Seit September 2020 bin ich wissenschaftlicher Mitarbeiter am Lehrstuhl für Wirtschaftsinformatik I (Prof. Dr. Pernul). Dort bin ich an einem vom Bundesministerium für Wirtschaft und Energie (BMWK) geförderten Forschungsprojekt beteiligt. \\\\Meine Forschungsthemen sind Cybersecurity und Datenanalysen im Zusammenhang mit dem Internet of Things (IoT).}


%---------------------------------------------------------------------------------------
%	WORK EXPERIENCE
%----------------------------------------------------------------------------------------

\vspace{10pt}
\cvsection{Berufserfahrung}
\vspace{4pt}

\cvevent
{02/2023 -- heute}
{Wissenschaftlicher Mitarbeiter}
{Lehrstuhl für Elektrische Energietechnik\newline FAU Erlangen-Nürnberg}
{Aufgabenbereiche: Technische Umsetzung der jeweiligen Arbeitspakete im Forschungsprojekt SISSeC und Lehre. }
\vfill\null

\cvevent
{03/2022 -- 09/2022}
{Masterand}
{\newline Vitesco Technologies Group AG}
{Aufgabenbereiche: Mitwirken im Forschungsprojekt SISSeC, das vom Bundesministeriums für Wirtschaft und Energie gefördert wird, Konzeption eines Puffer- und Persistenzsystems und Implementierung eines digitalen Zwillings. }
\vfill\null

\cvevent
{09/2019 - 02/2020}
{Tutor - IT-Security I}
{Lehrstuhl für Wirtschaftsinformatik I (Prof. Pernul)\newline Universität Regensburg}
{Kursinhalte: Live Hacking (SQL Injection), Klassische Kryptogaphie, Moderne Kryptogaphie (AES, DES), Kryptonanalyse, Identifikation und Authentifizierung, Rechte Management.}
\vfill\null

\cvevent
{03/2019 - 08/2019}
{Tutor - Informationssysteme: Entwicklungen und Trends}
{Lehrstuhl für Wirtschaftsinformatik I (Prof. Pernul)\newline Universität Regensburg}
{Kursinhalte: Einführung in PL/SQL, Prozeduren und Funktionen, Triggerprogrammierung, Objektrelationale Erweiterungen in PL/SQL, NoSQL Grundlagen.}
\vfill\null
\vfill\null

\cvevent
{10/2018 - 03/2019}
{Tutor - IT-Security I}
{Lehrstuhl für Wirtschaftsinformatik I (Prof. Pernul)\newline Universität Regensburg}
{Kursinhalte: Live Hacking (SQL Injection), Klassische Kryptogaphie, Moderne Kryptogaphie (AES, DES), Kryptonanalyse, Identifikation und Authentifizierung, Rechte Management.}
\vfill\null

\cvevent
{04/2018 - 10/2018}
{Nebenberufliche Wissenschaftliche Hilfskraft (NWHK)}
{Professur für Wirtschaftsinformatik (Prof. Schryen)\newline Universität Regensburg}
{Aufgabenbereiche: Unterstützung im EPIQUALIS (Epistomologische Fortschritte durch qualitative Literature-Reviews), Erstellung von Literaturdatensätze, Statische Datenanalyse, Forschungsdokumentation, Identifikation und Authentifizierung, Verwendung von Docker, Git, Python und R.}
\vfill\null

\cvevent
{10/2017 - 03/2018}
{Praktikant}
{Wertschöpfungsorientiertes Produktionssystem\newline BMW AG (Landshut)}
{Aufgabenbereiche: Weiterentwicklung der digitalen Prozesstafel, Mitverantwortung bei dessen Rollout, Implementierung einer Rüstungscheckliste, IST-Aufnahme der Taktzeiten von Prozessen.}
\vfill\null

\cvevent
{04/2017 - 10/2017}
{Studentische Hilfskraft (SHK)}
{Professur für Wirtschaftsinformatik (Prof. Schryen)\newline Universität Regensburg}
{Aufgabenbereiche: Unterstützung im EPIQUALIS (Epistomologische Fortschritte durch qualitative Literature-Reviews), Erstellung von Literaturdatensätze, Statische Datenanalyse, Forschungsdokumentation, Identifikation und Authentifizierung, Verwendung von Docker, Git, Python und R.}
\vfill\null

\cvevent
{10/2016 - 03/2017}
{Tutor - Theoretische Informatik}
{Professur für Wirtschaftsinformatik (Prof. Schryen)\newline Universität Regensburg}
{Kursinhalte: Einführung von Automaten, Deterministische endliche Automaten, Nichtdeterministische endliche Automaten, Reguläre Ausdrücke und Sprachen, Kontextfreie Grammatik und Sprachen, Push-down Automaten, Einführung von Turing Maschinen, Entscheidungsprobleme.}
\vfill\null

\cvevent
{03/2015 - 08/2016}
{Werkstudent}
{Manufacturing Executive Systems\newline DRÄXLMAIER Group (Vilsbiburg)}
{Aufgabenbereiche: Refactoring der DTO Model Klassen im MES/ PCM Software Projekt, Tests im Rahmen des MES Rollouts, Bearbeitung diverser Issues, Wiki.}
\vfill\null


\cvsection{Publikationen}
\begin{itemize}[leftmargin=*]
	\item Wagner, G. \& Empl, P. \&  Schryen G. (2020). "Designing a novel strategy for exploring literature corpora". In: \textit{Proceedings of the 28th European Conference on Information Systems (ECIS)}, Marrakesch, June 15-17, 2020.
\end{itemize}
% hofixes to create fake-space to ensure the whole height is used
\mbox{}
\vfill
\mbox{}
\vfill
\mbox{}
\vfill
\mbox{}
\vfill
\mbox{}
%\vfill
%\mbox{}
%\vfill
%\mbox{}


Erlangen, den \today     \hspace{1cm}   \hrulefill

\hspace*{30mm}\phantom{Erlangen, den \today }Alexander Pfannschmidt
